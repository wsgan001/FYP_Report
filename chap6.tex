\chapter{Conclusion}\label{Chap:6}

This paper presents an alternative approach for finding a shortest path in a time-dependent road network, based on GPS trajectories aggregated from thousands of taxis in Beijing, China. With the help of existing developer APIs from Baidu Maps, each GPS data point is mapped to a real road segment based on its longitude and lati\-tude. An innovative SOFM-based approach is proposed for identifying and removing outliers. The trajectories are then separated into individual trips and frequently vi\-sited streets in each trip are chosen as landmarks. Several time-dependent graphs with landmarks as vertices are constructed as an abstract representation of Beijing's road network. SOFM and Weibull Distribution are employed to estimate the travel time of each landmark graph edge. The estimates are compared with real-time estimates made by Baidu Maps. The worse-case RMSE is 87.96 seconds and the worst-case MER is -0.16, which indicates that this approach provides reasonable estimates as compared with Baidu's and has the potential for everyday use. A modified Dijkstra's algorithm is proposed for calculating shortest paths based on the estimates. 

This data-driven approach harvests the collective intelligence from thousands of taxi drivers who possess implicit knowledge about the time-dependent shortest paths in a road network. With more hand-hold GPS devices embedded in mobile phones available, it is possible in the future to incorporate personal driving trajectories into the travel time estimation, so that more personalised, accurate driving route suggestions can be provided. 