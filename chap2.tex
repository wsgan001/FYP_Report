%% This is an example first chapter.  You should put chapter/appendix that you
%% write into a separate file, and add a line \include{yourfilename} to
%% main.tex, where `yourfilename.tex' is the name of the chapter/appendix file.
%% You can process specific files by typing their names in at the 
%% \files=
%% prompt when you run the file main.tex through LaTeX.
\chapter{Data pre-processing}
\section{Data collection and preliminary cleaning}
The taxi GPS data used in this project is collected from the Computational Sensing Lab\cite{BPLL13} at Tsinghua University, Beijing, China. The data set contains approximately 83 million time-stamped taxi GPS records collected from 8,602 taxis in Beijing, from 1 May 2009 to 30 May 2009. The original data set consists of seven fields as shown in Table~\ref{Ta:orig_field}. ``WGS-84'' stands for ``World Geodetic System'' which is the reference coordinate system used by the GPS.

\begin{table}
\centering
\begin{tabular}{ | l | l | l | }
\hline
\textbf{Field} & \textbf{Explanation} \\ \hline
CUID & ID for each taxi \\ \hline
UNIX\_EPOCH & Unix timestamp in milliseconds since 1 January 1970\\ \hline
GPS\_LONG & Longitude encoded in WGS-84 multiplied by $10^{5}$\\ \hline
GPS\_LAT & Latitude encoded in WGS-84 multiplied by $10^{5}$ \\ \hline
HEAD & Heading direction in degrees with 0 denoting North\\ \hline
SPEED & Instantaneous speed in metres/second (m/s)\\ \hline
OCCUPIED & Binary indicator of whether the taxi is hired (1) or not (0)\\ \hline
\end{tabular}
\caption{Fields in the original data set}\label{Ta:orig_field}
\end{table}

The original data set comes in a binary file format. After the data is decoded and imported into a MySQL database, the first step in cleaning data is to delete all records with zero value in the SPEED field, since when a taxi is stationary it yields no valuable information about the \emph{trajectory} it is moving along. While being stationary could be due to a traffic jam, this kind of information is well captured by the time difference between the previous \emph{non-stationary} data point and the next \emph{non-stationary} data point. 

Moreover, all records must have a \emph{unique} pair of CUID and UNIX\_EPOCH fields, since it is not possible for a taxi to appear in two different places at the same moment in time. This is possibly due to some errors in aggregating the original data set.

\section{Reverse geo-encoding}
After the preliminary cleaning of the data set, the next step is to map each GPS data point to a road segment, which is also known as \emph{reverse geo-encoding}. A number of algorithms\cite{MAP09} have been proposed for that purpose, but most of them require an additional GIS\footnote{Geographic Information System} database of the road network in Beijing. This project adopts an alternative strategy which leverages on the existing public API\footnote{Application Programming Interface} for reverse geo-encoding. 

Currently, a number of online mapping platforms provide reverse geo-encoding services as part of their developer APIs. Amongst others, Google Maps and Baidu Maps offer relatively stable and fast reverse geo-encoding services. However, due to the ``China GPS shift problem''\cite{GSHF17} where coordinates encoded in WGS-84 format are required by regulations to be shifted by a large and variable amount when displayed on a street map, Google Maps is not able to display a GPS point correctly. 

Baidu Maps, on the other hand, has been using their own coordinate system called BD-09 which is an improved version of the Chinese official coordinate system, GCJ-02. Baidu provides a set of APIs to convert WGS-84-encoded coordinates into BD-09-encoded ones. 

In order to use Baidu APIs for reverse geo-encoding, the following system architecture has been set up as shown in Figure~\ref{Fig:archit}. The Apache HTTP server hides the MySQL database and sends HTTP POST request to Baidu Maps Web API to get converted coordinates. Then it updates the database through PHP \emph{mysqli} utilitity. 
\begin{figure}[h]
\includegraphics{architecture}
\centering
\caption{System architecture}\label{Fig:archit}
\end{figure}

After the conversion is completed, Baidu Maps API is used to reverse geo-encode all GPS data points.

\section{Related work}
The incentive for carrying out this project comes from a similar project\cite{TDR10}, and similar procedures are followed in this project but with some modifications. 


